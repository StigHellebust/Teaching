\documentclass[addpoints,12pt]{exam}

\usepackage{graphicx}
\usepackage[utf8]{inputenc}
\usepackage{mhchem}
%\usepackage{Exercise}

%\usetheme{JuanLesPins}
%\logo{\includegraphics[height=1cm]{../../graphics/crac}}
\setlength{\parskip}{1em}
\boxedpoints
\printanswers

\begin{document}

\begin{center}
\includegraphics{../../graphics/UCCchem}

\fbox{\fbox{\parbox{5.5in}{\centering
CM4019 Continuous Assessment Photochemistry\\Submission by September 28 2022, 10AM\\\vspace{1cm}This assessment is worth 10\% of the module mark}}}
\end{center}
\vspace{0.1in}
%\makebox[\textwidth]{Name and section:\enspace\hrulefill}
%\vspace{0.2in}
%\makebox[\textwidth]{Lecturer's name:\enspace\hrulefill}


\begin{questions}
%\pointname{}
\pointformat{\bfseries\boldmath[ \thepoints]}

\question[5]

A 100 cm\(^3\) vessel containing hydrogen and chlorine was irradiated with light of 400 nm. Measurements with a thermopile showed that \(11 \times 10^{-7}\) J of light energy was absorbed by the chlorine per second. During an irradiation of 1 minute the partial pressure of chlorine, as determined by the absorption of light and the application of Beer's law, decreased from 27.3 to 20.8 kPa (corrected to 0 \(^\circ\)C).

\begin{parts}
\part%[1]
What is the energy of each photon?

\begin{solution}
\(E=\frac{hc}{\lambda} = \frac{(6.626\times10^{-34}\ J\ s)(2.998\times10^8\ m\ s^{-1})}{400\times10^{-9}\ m} = 4.97\times 10^{-19}\ J\)
\end{solution}

\part%[2]
How many quanta were absorbed?

\begin{solution}
The number of quanta absorbed = \(\frac{(11\times10^{-7}\ J\ s^{-1})(60\ s)}{4.97\times10^{-19}\ J} = 1.33\times10^{14}\)
\end{solution}

\part%[3]
How many moles of Chlorine reacted? 

\begin{solution}
Since \(PV = nRT,\ V\Delta P = RT\Delta n\)

\(\Delta n = \frac{V \Delta P}{RT} = \frac{(0.1\times10^{-3}\ m^3)[(27.3 - 20.8)\times10^3\ Pa](6.02\times10^{23}\ mol^{-1})}{8.314\ J\ K^{-1}\ mol^{-1})(273\ K)}= 1.72\times10^{20}\) molecules

\(=2.86\times10^{-4}\) mol
\end{solution}

\part%[1]
What was the quantum yield?

\begin{solution}
Since \ce{H2 + Cl2 = 2HCl}

\(\phi = \frac{1.72\times10^{20}}{1.33\times10^{14}}\times2 = 2.6\times10^6\)
\end{solution}

\end{parts}

\question[3] 

You want to use photochemical reactions to produce some product. Assume that 5\% of the electric energy consumed by a quartz-mercury-vapour lamp goes into light, and 30\% of this is photochemically effective.

The electricity costs 5 cents per kilowatt-hour, you want to produce 1 lb (453.6 g) of an organic compound having a molar mass of 100g/mol and the average effective wavelength is 400 nm, The reaction has a quantum yield of 0.8 molecule per photon.

\begin{parts}

\part%[1]
Calculate the number of quanta required

\begin{solution}
Number of quanta required \(=\frac{(453.6\ g)(6.022\times10^{23}\ mol^{-1})}{100\ g\ mol^{-1})(0.8)}=3.41\times10^{24}\)
\end{solution}

\part%[3]
Calculate the energy required

\begin{solution}
The energy per photon: \(E=\frac{hc}{\lambda} = \frac{(6.26\times10^{-34}\ J\ s)(2.998\times10^8\ m\ s^{-1})}{400\times10^{-10}\ m} = 4.97\times10^{-19}\ J\)

The total energy required = \(\frac{(3.41\times10^{24})(4.97\times10^{-19}\ J)}{(0.05)(0.3)} = 1.13\times10^8\ J\)

\end{solution}

\part%[1]
Calculate the cost to produce 453.6 g of the compound

\begin{solution}
1 kW hr = \((10^3\ J\ s^{-1})(60\times60s) = 36\times10^5\ J\)

\(Cost = \frac{1.13\times10^8\ J)(EURO 0.05 /KWhr)}{36\times10^5\ J/KWhr} = EURO 1.6\)
\end{solution}

\end{parts}

\question[2] 

The quantum yield is 2 for the photolysis of gaseous \ce{HI} to \ce{H2 + I2} by light of 253.7 nm wavelength. Calculate the number of moles of \ce{HI} that will be decomposed if 300 J of light of this wavelength is absorbed.

\begin{solution}
\(E=\frac{hc}{\lambda} = \frac{(6.626\times10^{-34}\ J\ s)(3\times10^8\ m\ s^{-1})}{253.7\times10^{-9}\ m)} = 7.84\times10^{-19}\ J\)

\(n=\frac{(2)(300\ J)}{(7.84\times10^{-19}\ J)(6.02\times10^{23}\ mol^{-1})}=1.27\times10^{-3}\) mol of HI
\end{solution}

\end{questions}

\end{document}