\documentclass[addpoints,12pt]{exam}

\usepackage{graphicx}
\usepackage[utf8]{inputenc}
\usepackage{pdfpages}
\usepackage{textcomp}
%\usepackage{fancyhdr,url}
\usepackage{amsmath}
\usepackage{exercise}
\usepackage{tikz}
\usetikzlibrary{datavisualization,shapes,arrows}
\usepackage{smartdiagram}
\usepackage[version=4]{mhchem}

\setlength{\parskip}{1em}
\boxedpoints
\printanswers

\begin{document}

\begin{center}
\includegraphics{../../../graphics/UCCchem}

\fbox{\fbox{\parbox{5.5in}{\centering
CM3104 Aquatic chemistry\\
\center{\tiny{version 2020.01.1}}}}}
\end{center}
\vspace{0.1in}

\begin{questions}
\question
If the concentration of atmospheric CO\textsubscript{2}  were to double from the current value of 400 ppm, what would be the calculated pH of rainwater (assuming CO\textsubscript{2}  were the only acidic input)? With respect to rising CO\textsubscript{2}  levels, do we have to be concerned about enhanced aciditiy of rain in addition to potential climate warming? (hint: notes section 1.3)

\begin{solution} {\color{red}
This is similar to the example in Lecture notes page 7. Carbon dioxide dissolves in water in accordance with Henry's law. 

\ce{CO2 + H2O -> H2CO3}, \quad with an equilibrium constant \(K_s = \frac{[H_2CO_3]}{P_{CO_2}} = 10^{-1.5}\) M/atm.

If the atmospheric \ce{CO2} is at 800 ppm then \(P_{CO_2} = 800 \times 10^{-6}\) atm.

\([H_2CO_3] = K_sP_{CO_2} = 10^{-1.5} \times 800 \times 10^{-6} = 2.53 \times 10^{-5}\)

\ce{H2CO3 = H+ + HCO3-}, \(K_a = 10^{-6.4}\), so we say \([H^+] \approx [HCO_3^-]\) and

\([H^+]^2 = K_a[H_2CO_3] = 10^{-6.4} \times 2.53\times 10^{-5} = 1.007\times10^{-11} \implies [H^+] = 3.173\times 10^{-6} \implies\) pH = 5.5}
\end{solution}

\question
How much \ce{Fe2+} could be present in water containing \(1.0 \times 10^{-2}\) M \ce{HCO3-} without causing precipitation of \ce{FeCO3} (\(K_{sp} = 10^{-10.7}\))?


\begin{solution} {\color{red} This is similar to the example on page 16-17 in the notes. The reaction is \ce{FeCO3 + H2CO3 = Fe2^+ = 2HCO3-}
 
 So the equilibrium is \(K = \frac{[Fe^{2+}][HCO_3^-]^2}{[H_2CO_3]} = \frac{K_{sp}K_{a1}}{K_{a2}}\)
 
 The concentration of \ce{H2CO3} is given by the atmospheric level (so we'll use \(10^{-4.9}\) as we worked out before)
 
 because we can say \([HCO_3^-] \approx 2[Fe^{2+}]\) we get
 
 \(K = \frac{4[Fe^{2+}]^3}{[H_2CO_3]} \implies [Fe^{2+}] = \left(\frac{\frac{K_{sp}K_{a1}}{K_{a2}}\times 10^{-4.9}}{4}\right)^{1/3}=8.11\times10^{-5}\)
 }
 \end{solution}
 
\question
Describe the three major buffer ranges for neutralizing acidic inputs to soils. For each of the ranges, include in your description: (1) the pH range over which the buffer operates; (2) the major chemical component(s) that participates in the buffering reactions; and (3) the chemical reaction by which H+ is neutralised

\begin{solution} {\color{red}
Buffering ranges are described in sections 3-5 in the notes, see figure 6 on page 21 for summary of the buffering ranges.
}
\end{solution}

\question
Why is the process of acid deposition beneficial from the point of view of air quality? How does this process transfer a short-term air-pollution problem into long-term problems of soil and water pollution?

\begin{solution} {\color{red}
Think about the example of Big Moose Lake, which was getting acidified because of emissions of \ce{SO2} to the atmosphere. Because the \ce{SO2} was being removed from the air as acid rain, it didn't build up as a permanent problem of air quality, but transferred the problem to soil and water
}
\end{solution}

\question
a) The total amount of base in a water sample can be determined by titration with standard acid, and is usually reported as the alkalinity, in equivalents of acid per liter (eq/L). If the alkalinity of a carbonate-containing sample is \(2.0 \times 10^{-3}\) eq/L, and the pH is 7.0, what are the concentrations of \ce{OH-}, \ce{CO_3^{2-}}, \ce{HCO3-} and \ce{H2CO3}?
 

b) How do these concentrations change if the pH of the sample increases to 10.0 as a result of photosynthesis by algae (for simplicity, assume no input of CO\textsubscript{2} from the atmosphere) What weight of biomass, (\ce{CH2O}), is produced?

\begin{solution} {\color{red}
alkalinity is given by (page 17 of notes):

a) \(alkalinity = [HCO_3^-] + 2[CO_3^{2-}] + [OH^-] - [H_3O^+]\)

Since pH = 7, that means that \([H_3O^+] = [OH^-] = 10^{-7}\), so the last two terms cancel each other. That means \([HCO_3^-] + 2[CO_3^{2-}] = 2.0 \times 10^{-3}\) eq/L

From the equilibrium we also have that \([HCO_3^-] = \frac{[H^+][CO_3^{2-}]}{K_{a2}}\). We also know that \([H^+] = 10^{-7}\), so

\(2.0\times10^{-3} - 2[CO_3^{2-}] = \frac{[H^+][CO_3^{2-}]}{K_{a2}}\)

Solve that and we get \([CO_3^{2-}] = 9.35\times10^{-7}\)

\medskip b) if photosynthesis removes enough \ce{CO2} to increase the pH to 10, (and without more \ce{CO2} being dissolved, it means the equilibrium is shifted towards the undissociated form, which releases \ce{CO2} into the water \ce{H2CO3 -> CO2 + H2O}.

if the Ph changes from 7 to 10 that means that the concentration of \(H^+\) changes from \(10^{-7}\) to \(10^{-10}\), so \(10^{-7} - 10^{-10} = 9.99\times10^{-8}\) moles of \ce{H+} have reacted, per litre. for each mole of \ce{H^+} that is consumed in the reaction \ce{H+ + HCO3- -> H2CO3 -> CO2 + H2O}, we release one mole of \ce{CO2} for photosynthesis. so the weight of biomass is \(9.99\times10^{-8}\) mol/L \(\times\) 30 g/mol \(= 2.997\times10^{-6}\) g/L
}
\end{solution}

\end{questions}

\end{document}
